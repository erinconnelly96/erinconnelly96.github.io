\documentclass[margin,line,pifont,palatino,courier]{res}
\usepackage[utf8]{inputenc}
\newcommand{\D}{\textbf{D.}}
\usepackage{pifont}
%\usepackage[latin1] { inputenc}
\usepackage{hyperref}
%\topmargin .5in
%\oddsidemargin -.5in
%\evensidemargin -.5in
%\textwidth=6.0in
 \textheight=9.0in
%\itemsep=0in
%\parsep=0in
\usepackage{fancyhdr}
%\topmargin=0in
%\textheight=8.5in
\pagestyle{fancy}
\renewcommand{\headrulewidth}{0pt}
\fancyhf{}
%\cfoot{\thepage}
%\lfoot{\textit{\footnotesize Research Statement}}
\rfoot{{\footnotesize Timothy Duff, \thepage}}


\newenvironment{list1}{
  \begin{list}{\ding{113}}{%
      \setlength{\itemsep}{0in}
      \setlength{\parsep}{0in} \setlength{\parskip}{0in}
      \setlength{\topsep}{0in} \setlength{\partopsep}{0in}
      \setlength{\leftmargin}{0.17in}}}{\end{list}}
\newenvironment{list2}{
  \begin{list}{$\bullet$}{%
      \setlength{\itemsep}{0in}
      \setlength{\parsep}{0in} \setlength{\parskip}{0in}
      \setlength{\topsep}{0in} \setlength{\partopsep}{0in}
      \setlength{\leftmargin}{0.2in}}}{\end{list}}

\begin{document}

\name{Timothy Duff \vspace*{.1in}}

\begin{resume}

\section{\sc}

\vspace{.05in}
\begin{tabular}{@{}p{2.05in}p{2.95in}}
Department of Mathematics  & \verb+timduff@uw.edu+\\
University of Washington & office phone: +1 206 543 6183\\
Padelford Hall PDL C-505 & cell: +1 813 503 1570 \\
Seattle WA 98195-4350
\end{tabular}

\section{\sc Employment}

University of Washington Department of Mathematics \hfill Seattle, WA\\
2021--2024\\
NSF Postdoctoral Fellow\\
Mentor: Rekha Thomas


\section{\sc Education}

Georgia Institute of Technology \hfill Atlanta, GA\\
2016--2021\\
Ph.D. in Algorithms, Combinatorics, and Optimization\\
Dissertation: Applications of monodromy in solving polynomial systems\\
Advisor: Anton Leykin\\[1em]
University of Oxford \hfill Oxford, UK \\
\hfill 2014--2015\\
M.Sc.~\emph{with distinction} in Mathematics and Foundations of Computer Science\\[1em]
New College of Florida\hfill Sarasota, FL\\
2010--2014\\
B.A.~with Concentration in Mathematics

\section{\sc Honors, Awards, and Funding}

\begin{tabular}{@{}p{0.2in}p{5in}}
  2023 & NSF Conference Grant (Macaulay2 Week at UMN)\\
    $\rightarrow $ & co-PI on DMS 2302476, \$48,132,
                     Summer 2023\\
  2022 & Best paper, Computer Vision and Pattern Recognition (CVPR) \\
  $\rightarrow $   & 1 awarded out of 2067 papers accepted / 8161 submitted\\
& Conference ranked 4th among all publications by \href{https://scholar.google.com/citations?view_op=metrics_intro&hl=en}{Google Scholar}\\
2022 & Georgia Tech School of Mathematics Best PhD Thesis Award\\
2021 & Georgia Tech School of Mathematics Top Graduate Student Award\\
  2021 & NSF Mathematical Sciences Postdoctoral Research Fellowship\\
  $\rightarrow $ & PI on DMS 2103310, \$150,000, 2021--2024\\
2020 & Georgia Tech Algorithms and Randomness Center Fellowship\\
2019 & Best student paper, International Conference on Computer Vision (ICCV) \\
$\rightarrow $   & 1 awarded out of 1075 papers accepted / 4303 submitted\\
2016 & Georgia Tech Presidents Fellowship\\
2014 & University of Oxford Frost Scholarship\\
2014 & Outstanding poster award, Joint Math Meetings\\
2013 & Barry Goldwater Scholarship
\end{tabular}

\section{\sc Publications}

Except where marked (*), authors are listed alphabetically.

(*) \textit{PLMP - Point-Line Minimal Problems in Complete Multi-View Visibility}
(\D, Kathlén Kohn, Anton Leykin, Tomas Pajdla.)
To appear in Transactions on Pattern Analysis and Machine
Intelligence, 2023. Conference version: Proc.~ICCV 2019.


(*) \textit{Learning to solve hard minimal problems}
(Petr Hruby, \D, Anton Leykin, Tomas Pajdla.)
To appear in Transactions on Pattern Analysis and Machine Intelligence, 2023. Conference version: Proceedings of CVPR 2022.


\textit{u-generation: solving systems of polynomials equation-by-equation}
(Jose Israel Rodriguez, \D, Anton Leykin.)
Numerical Algorithms, 2023, pp.~1-26.

(*) \textit{Four-view geometry with unknown radial distortion}
(Petr Hruby, Viktor Korotynskiy, \D, Luke Oeding, Marc Pollefeys, Tomas Pajdla, Viktor Larsson.)
CVPR 2023.

\textit{Using monodromy to recover symmetries of polynomial systems}
(\D, Viktor Korotynskiy, Tomas Pajdla, Margaret Regan.)
Proceedings of ISSAC 2023. Association for Computing Machinery, New York, NY, USA, 251–259.

(*) \textit{Trifocal Relative Pose from Lines at Points}
    (Ricardo Fabbri, \D, Hongyi Fan, Margaret Regan, David da Costa de Pinho, Elias Tsigaridas, Charles Wampler, Jonathan Hauenstein, Benjamin Kimia, Anton Leykin, Tomas Pajdla.)\\
    Transactions on Pattern Analysis and Machine Intelligence, 2023.\\
    $\rightarrow $ Conference version: Proceedings of CVPR 2020.

\textit{Polyhedral homotopies in Cox coordinates}
(\D, Simon Telen, Elise Walker, Thomas Yahl.) 
Journal of Algebra and its Applications, 2023.

\textit{Galois/monodromy groups for decomposing minimal problems in 3D reconstruction}
(\D, Viktor Korotynskiy, Tomas Pajdla, Margaret Regan.) 
SIAM Journal on Applied Algebra and Geometry, 2023, 6(4), 740-772.

\textit{An Atlas for the Pinhole Camera}
(Sameer Agarwal, \D, Max Lieblich, Rekha Thomas.
Foundations of Computational Mathematics, 2023.

\textit{Signatures of Algebraic Curves via Numerical Algebraic Geoemtry}
(\D, Michael Ruddy.) 
Journal of Symbolic Computation, 2023, 115, pp.452-477.\\
$\rightarrow $ Conference Version: Proceedings ISSAC 2020.


\textit{Nonlinear Algebra and Applications }
(Paul Breiding, Türkü Özlüm Çelik, \D, Alexander Heaton, Aida Maraj, Anna-Laura Sattelberger, Lorenzo Venturello, Oğuzhan Yürük.)
Nonlinear Control and Algebra, 2021.

\textit{PL$_1$P---Point-line minimal problems under partial visibility in three views}
(\D, Kathlén Kohn, Anton Leykin, Tomas Pajdla.) Accepted to International Journal of Computer Vision.
Conference version: ECCV 2020.

    \textit{Certification for polynomial systems via square subsystems}
    (\D, Nickolas Hein and Frank Sottile.)
    Journal of Symbolic Computation. Extended abstract presented at MEGA (Effective Methods in Algebraic Geometry) 2019.

    \textit{Monodromy solver: sequential and parallel}
    (Nathan Bliss, \D, Anton Leykin, Jeff Sommars.)
    Proceedings of ISSAC 2018.
    
    \textit{Solving polynomial systems via homotopy continuation and monodromy}
    (\D, Cvetelina Hill, Anders Jensen, Kisun Lee, Anton Leykin, Jeff Sommars.)
    IMA Journal of Numerical Analysis, 39(3), 1421--1446, 2018.

    \textit{Polynomial automata: Zeroness and applications}
    (Micheal Benedikt, \D, Aditya Sharad, James Worrell.)
    Proc.~LICS 2017 (ACM/IEEE Symp.~on Logic in Computer Science).
    
    \textit{Robust graph ideals}
    (Adam Boocher, Bryan Brown, \D, Laura Lyman, Takumi Murayama, Amy Nesky, Karl Schaefer.)
    Annals of Combinatorics, 2015.

\section{\sc Preprints / Under review}

\textit{Algebra \& Geometry of Camera Resectioning} (Erin Connelly,
\D, Jessie Loucks-Tavitas.)
Submitted to Journal of Algebra.

\textit{Subalgebrabases in Macaulay2} (Michael Burr, \D, Oliver
Clarke, Nathan Nichols, Elise Walker.) Under revision at Journal of Software for Algebra and Geometry.

\textit{Line Multiview Ideals}
(w/ Paul Breiding, \D, Lukas Gustafsson, Felix Rydell, Elima Shehu.) Submitted to Communications in Algebra

\textit{Geometric Initial Orbit Determination Using Bearing Measurements}
(\D, Michela Mancini, Anton Leykin, John Christian.)
Presented at the 2022 AIAA Astrodynamics Specialist Conference.
Under review at Journal of Astronautical Sciences.


\section{\sc Organization and Mentoring}

\begin{tabular}{@{}p{1.4in}p{5.5in}}
  Jan 2024 & Organizing \emph{Mathematics of Computer Vision} Special Session at \hyperlink{https://www.jointmathematicsmeetings.org//jmm}{JMM 2024} \\
  July 2023 & Organized \emph{Algebraic Vision} and \emph{SAGBI/Khovanskii Bases} Minisymposia\\
  &at \hyperlink{https://www.siam.org/conferences/cm/conference/ag23}{2023 SIAM Conference on Applied Algebraic Geometry}\\
  March--May 2023 &
                    \hyperlink{https://wxml.math.washington.edu/}{WXML}
                    Group Leader: group topic \\
&``Generalized Matrix Nearness and Homotopy Continuation
''\\
  June 2023 & Organized \hyperlink{https://macaulay2.github.io/Workshop-2023-Minneapolis/}{Macaulay2 Minischool and Workshop}\\
  Jan 2023 & Organized \hyperlink{https://www.jointmathematicsmeetings.org//meetings/national/jmm2023/2270_shortcourse}{AMS Short Course} on \\& \indent ``Polynomial systems, homotopy continuation, and applications''\\
  Jan 2023 & Organized \hyperlink{https://www.jointmathematicsmeetings.org/meetings/national/jmm2023/2270_program.html}{JMM 2023} Special Session on \\&\indent ``Polynomial systems, homotopy continuation, and applications''\\
  Feb 2022 &  Organized Algberaic Vision Network virtual meeting\\
  Jan--March 2022 & WXML Group Leader: group topic \\
  & ``Resectioning and Computational Algebriac Geometry''\\
  Aug 16--20 2021& Organized \emph{Algebraic Vision} Minisymposium \\
  &at \hyperlink{https://www.siam.org/conferences/cm/conference/ag21}{2021 SIAM Conference on Applied Algebraic Geometry}\\
  Summer 2020 & Group leader in \hyperlink{https://warwick.ac.uk/fac/sci/maths/research/events/2019-20/m2/}{Warwick Macaulay2 workshop}\\
  Fall 2019--2020 & Mentor in GA Tech Directed Reading program\\
  Spring--Summer 2018 & AMS Club secretary (organized tutorials and social activities)\\
  Fall 2017                & Organized student Algebraic Geometry seminar, GA Tech\\
  Fall 2016 -- Spring 2018 & Organized graduate Research Horizons seminar, GA Tech
\end{tabular}  



\section{\sc Other Professional Service}

I have served as a referee for the following journals: AIMS Mathematics, AMS Proceedings and Symposia in Applied Mathematics, ASME Journal of Computing and Information Science in Engineering, International Journal of Computer Vision, Journal of Symbolic Computation, La Matematica, Numerical Algorithms, ACM Journal of Pattern Recognition, SIAM Journal on Applied Algebra and Geometry, SIGMA (Symmetry, Integrability, and Geometry, Methods and Applications)

I have refereed papers for the following conferences: CVPR, ISSAC, MEGA, NeuRIPS.

I have authored several reviews for zbMath (formerly Zentralblatt.)

I am serving as publicity chair and webmaster for \hyperlink{https://www.issac-conference.org/2024/}{ISSAC 2024}.

\section{\sc Extended Professional Travel}

\begin{tabular}{@{}p{0.9in}p{0.2in}p{4.2in}}
  Jan 30--Mar 1 & 2020 & Max Planck Institute for Mathematics in the Sciences (Leipzig, DE)\\
  Jan 27--Feb 15 & 2019 & ICERM Algebraic Vision Research Cluster (Providence, RI)\\
  Sep 1--Dec 1 & 2018 & ICERM Nonlinear Algebra Semester Program (Providence, RI)
\end{tabular}

\pagebreak

\section{\sc Teaching Experience}

At University of Washington:

\begin{tabular}{@{}p{0.4in}p{0.3in}p{4in}}
  Winter & 2024 & Instructor, Advanced Linear Algebra (Math 318, est.~120 students)\\
  Fall   & 2023 & Instructor, Matrix Algebra (Math 208, est.~60 students)\\
  Winter & 2023 & Instructor, Special Topics: Introduction to Algebraic Computation (Math 582, 10 students)\\
  Fall   & 2022 & Instructor, Modern Algebra (Math 402, 60 students)
\end{tabular}

At Georgia Tech:

\begin{tabular}{@{}p{0.4in}p{0.3in}p{4in}}
  Summer   & 2021 & TA, Differential Equations (Math 2552, 1 section, online)\\
  Fall   & 2020 & TA, Linear Algebra (Math 1554, 2 sections, online)\\
  Summer & 2020 & TA, Survey of Calculus (Math 1712, online)\\
  Fall   & 2019  & TA, Differential Equations (Math 2552, 2 sections)\\
  Spring &  2019 & Grader, Statistical Theory (Math 3236)\\
  Spring & 2019 & Grader, Abstract Algebra II (Math 4108)\\
  Spring & 2019 & Lecture Assistant, Applied Combinatorics (Math 3012, 2 sections)\\
  Summer & 2018 & TA, Linear Algebra (Math 1554)\\
  Spring & 2018 & TA, Multivariable Calculus (Math 2551, 2 sections)\\
  Fall & 2017 & TA,  Discrete Mathematics (Math 2663, 2 sections)\\
  Summer & 2017 & Lecture Assistant, Differential Calculus (Math 1551)\\
  Spring & 2017 & TA, Linear Algebra (Math 1554)\\
  Spring & 2017 & TA, Finite Mathematics (Math 1711)\\
  Spring & 2016 & TA, Finite Mathematics (Math 1711)
\end{tabular}
  

    
\section{\sc Talks}    

\par Joint CUNY/NYU Courant/NCSU Seminar
\hfill October 23, 2023, virtual\\
\textit{Geometry of 2, 3, or 4 Cameras}. Seminar in Symbolic-Numeric Computing.

\par Texas A\&M University  \hfill September 22, 2023, College Station
TX\\
\textit{Geometry of 2, 3, or 4 Cameras}. Geometry Seminar.

\par TU Eindhoven  \hfill July 13, 2023, Eindhoven NL\\
\textit{Structured Polynomial Systems in Applications: Angles-only Orbit Determination and Radial Camera Relative Pose}. SIAM Conference on Applied Algebra and Geometry.

\par University of Wisconsin Madison \hfill May 10, 2023, Madison WI\\
\textit{Geometry of 2, 3, or 4 cameras}. SILO (Systems/Information/Optimization/Learning.)

\par University of Wisconsin Madison \hfill May 8, 2023, Madison WI\\
\textit{Tutorial on Numerical Algebraic Geometry}. MAFA Seminar.

\par Universita di Trento \hfill May 4, 2023, Trento IT (virtual)\\
\textit{Solving camera relative pose problems with homotopy continuation}. Seminar on Geometry and Topology for Data Analysis.

\par Georgia Institute of Technology. \hfill April 16, 2023, Atlanta GA.\\
\textit{Computing 28 bitangents to a plane quartic.} Meeting on Applied Algebraic Geometry.

\par University of Washington. \hfill April 25, 2023, Seattle WA.\\
\textit{Relative Pose Problems and their Branched Covers.} Algebraic Geometry Seminar.

\par Georgia Institute of Technology. \hfill April 16, 2023, Atlanta GA.\\
\textit{Computing 28 bitangents to a plane quartic.} Meeting on Applied Algebraic Geometry.

\par Georgia Institute of Technology. \hfill Mar 18, 2023, Atlanta, GA.\\
\textit{Galois groups, radial quadrifocal tensors, and principal minors.} Southeastern AMS Sectional.

\par Georgia Institute of Technology. \hfill Mar 9, 2023, Atlanta, GA.\\
\textit{Geometry of two, three, or four cameras.} ACORN (Algorithms, Combinatorics, and Optimization Research Network) Meeting.

\par Pacific Northwest National Lab.\hfill Mar 3, 2023, Seattle, WA (virtual.)\\
\textit{Geometry of two, three, or four cameras.} Topology, Algebra, and Geometry in the Mathematics of Data Science (TAG-DS) seminar.

\par Joint Mathematics Meetings \hfill Jan 4, 2023, Boston, MA.\\
\textit{Learning to Solve Hard Minimal Problems.} Special Session on Topology, Algebra, and Geometry in the Mathematics of Data Science (TAG-DS.)

\par Joint Mathematics Meetings \hfill Jan 4, 2023, Boston, MA.\\
\textit{An Atlas for the Pinhole Camera.} Special Session on Applied Enumerative Geometry.

\par Boise State University \hfill Oct 7, 2022, Boise, ID (online)\\
\textit{An Atlas for the Pinhole Camera} Topics in Algebra, Topology, Etc. (TATERS) Research Seminar

\par Carnegie Mellon University \hfill Sep 15, 2022, Pittsburgh, PA (online)\\
\textit{An Atlas for the Pinhole Camera} Optimization, Algebra, and Geometry Seminar

\par AAS/AIAA Astrodynamics Specialist Conference \hfill Aug 7--11 2022, Charlotte NC\\
\textit{Geometric Initial Orbit Determination from Bearing Measurements.} 

\par University of Texas at Austin. \hfill June 13, 2022, Austin, TX\\
\textit{Structured polynomial constraints in computer vision.} Data and Algebra seminar.

\par Czech Institute of Informatics, Robotics, and Cybernetics \hfill June 3, 2022, Prague, CZ\\
\textit{Polynomial and constraints on points and cameras.} IMPACT/AAG seminar. 

\par Czech Institute of Informatics, Robotics, and Cybernetics \hfill May 20, 2022, Prague, CZ\\
\textit{Galois/monodromy groups for decomposing minimal problems in 3D reconstruction.} IMPACT/AAG seminar. 

\par Joint Math Meetings \hfill April 6--9, 2021, Washington, DC (online)\\
\textit{Galois groups in 3D reconstruction} AMS Special Session on Structured Polynomial Systems In Mathematics and Its Applications

\par GA Tech Algebra Seminar \hfill March 29, 2022, Atlanta, GA\\
\textit{Image formation ideals}

\par SIAM Applied Algebraic Geometry Conference. \hfill Aug 16--20 2021, College Station, TX (online)\\
\textit{Numerical algebraic geometry meets differential invariants} Minisymposium on new trends in polynomial system solving

\par Workshop on Software and Applications of Numerical Nonlinear Algebra  \hfill May 31--June 2 2021, Leipzig DE (online)\\
\textit{Tale of two homographies}

\par Joint SIAM Student Conference (Southeast) \hfill April 3, 2021 (online)\\
\textit{Tale of two homographies}

\par Joint Math Meetings \hfill January 6--9, 2021, Washington, DC (online)\\
\textit{Galois groups in 3D reconstruction} AMS Special Session on Numerical Methods for Solving Polynomial Systems

\par SIAM Texas / Louisiana Section Annuel Meeting \hfill October 17, 2020, College Station TX (online).\\
\textit{Galois groups in 3D reconstruction} Part of a minisymposium on applications of algebraic geometry

\par Texas A \& M Geometry seminar (online) \hfill September 28, 2020, College Station TX (online)\\
\textit{Galois groups of structured polynomial systems}

\par ICERM Workshop on Galois and monodromy groups in applications \hfill Aug 28--Sep 2, 2020, Providence, RI (online)\\
\textit{Galois groups of minimal problems.} Part of a ``Hot topics" series

\par ECCV 2020 \hfill August 23--27, 2020, Glasgow, UK (online)\\
\textit{Point-line minimal problems with partial visibility} Main papers session.

\par ISSAC 2020 \hfill July 20-22, 2020, Kalamata, GR (online)\\
\textit{Numerical equality tests for rational maps and signatures of curves.} Main papers session.

\par SIAM Mathematics of Data Science Conference \hfill June 11-12, 2020, Cleveland OH (online)\\
\textit{Minimal problems with missing data.} Part of a minisymposium on Algebraic Geometry and Machine Learning.

\par Max Planck Institute for Mathematics in the Sciences \hfill Feb 28, 2020, Leipzig, DE\\
\textit{Finding, solving, and simplifying minimal problems in computer vision.} Nonlinear algebra seminar

\par ICCV 2019 \hfill Oct 27--Nov 2, 2019, Seoul\\
\textit{Point-line minimal problems in complete multi-view visibility.} Award papers session, Geometric multiview geometry oral session, and poster session.

\par AMS Fall Northwestern Sectional Meeting \hfill September 14--15, 2019, Madison WI\\
\textit{Certification for polyomial systems via square subsystems.} Special Session on Applied Algebra.

\par Czech Institute of Informatics, Robotics, and Cybernetics \hfill July 18, 2019, Prague, CZ\\
\textit{Intro to homotopy continuation with a view towards minimal problems.} IMPACT/AAG seminar. 

\par SIAM Applied Algebraic Geometry Conference. \hfill July 9--13, Universitet Bern, Bern, SZ \\
\textit{Certification for polyomial systems via square subsystems.} Minisymposium on Algebraic Methods for Polynomial System Solving.

\par MEGA \hfill June 17--21, 2019, University Complutense of Madrid, Madrid, ES\\
\textit{Certification for polynomial systems via square subsystems.}

\par AMS Spring Southeastern Sectional Meeting. \hfill March 15--17, 2019, Auburn, AL.\\
\textit{Certification for polyomial systems via square subsystems.} Special Session on Applications of Algebraic Geometry.

\par AMS Fall Southeastern Sectional Meeting. \hfill November 3--4, 2018 Fayetteville, AR.\\
\textit{Monodromy solver: sequential and parallel.} Special Session on Numerical methods for nonlinear equations.

\par SIAM Annual Meeting. \hfill Jul 9-13, 2018, Portland, OR. \\
\textit{Randomized Aspects of Polynomial System Solving.} Minisymposium on Numerical Algebraic Geometry.

\par Max Planck Institute for Mathematics in the Sciences \hfill Jun 5, 2018, Leipzig, DE\\
\textit{Monodromy solver and 27 lines on a cubic} Macaulay2 workshop

\par SIAM Applied Algebraic Geometry Conference. \hfill Sep 30--Aug 4, 2017, Atlanta, GA. \\
\textit{Solving polynomial systems via homotopy continuation and monodromy.} Minisymposium on Theoretical Advances in Numerical Algebraic Geometry.

\par AMS Western Sectional Meeting. \hfill October 8--9, 2016 Denver, CO.\\
Joint talk with K.~Lee \textit{Solving polynomial systems via homotopy continuation and monodromy.} Thematic Program on Foundations of Numerical Algebraic Geometry.

\par Joint Meetings of the AMS and MAA. \hfill January 15--18, 2014, Baltimore, MD.\\
Poster presentation: \textit{Robust Graph Ideals.} Presented jointly w/ K. Schaefer.

\par Berkeley/Stanford Joint REU Conference. \hfill July 31st, 2013, Palo Alto, CA.\\
\textit{Geometric Invariants on Monomial Curves.} w/ Takumi Murayama, Karl Schaefer. 

\end{resume}

\end{document}
